\documentclass{beamer}
\setbeamercovered{transparent=25} 
\usepackage[utf8]{inputenc}
\usepackage{amsfonts,amsmath,oldgerm}
\usefonttheme[onlymath]{serif}
\usepackage{tikz}
\usepackage{xcolor}

\newcommand{\greencheck}{\color{green}\checkmark}
\newcommand{\redcross}{\color{red}\times}
\newcommand{\tr}{\text{tr}}

\newcommand{\Tr}{\text{Tr}}

\newcommand{\mb}{\mathbf}

\newcommand{\bsy}{\boldsymbol}

\newcommand{\tb}{\textbf}

\newcommand{\ti}{\textit}

\newcommand{\btheta}{\boldsymbol{\theta}}

\newcommand{\brho}{\boldsymbol{\rho}}

\newcommand{\nmeasn}[1]{$n_{\text{meas}}=#1$}

\newcommand{\nitern}[1]{$n_{\text{iter}}=#1$}

\newcommand{\nburninn}[1]{$n_{\text{burnin}}=#1$}

\newcommand{\rhorankn}[1]{\text{rank}$(\rho)=#1$}

\newcommand{\nmeas}[0]{$n_{\text{meas}} $ }

\newcommand{\niter}[0]{$n_{\text{iter}} $ }

\newcommand{\nburnin}[0]{$n_{\text{burnin}} $ }

\newcommand{\rhorank}[0]{\text{rank}$(\rho) $ }

\newcommand{\semitransp}[2][35]{\textcolor{fg!#1}{#2}}

\usetheme{default}
\addtobeamertemplate{navigation symbols}{}{%
    \usebeamerfont{footline}%
    \usebeamercolor[fg]{footline}%
    \hspace{1em}%
    \insertframenumber/\inserttotalframenumber
}
\title{Master's thesis: Numerical comparison of MCMC methods for Quantum Tomography}
\author[Mokeev]
{ Danila Mokeev\\{\small Supervisors: Estelle Massart and Andrew Thompson}}
\institute[EPL]{Ecole Polytechnique de Louvain}
\date{21st of June 2024}

\AtBeginSection[]
{
  \begin{frame}
    \frametitle{Table of Contents}
    \tableofcontents[currentsection]
  \end{frame}
}

\begin{document}

\frame{\titlepage}

\begin{frame}
\frametitle{Table of Contents}
\tableofcontents
\end{frame}


\begin{frame}{Problem: quantum state reconstruction}
    \tb{Goal}: We want to reconstitute a quantum state\medbreak
    Unfortunately, there are some challenges: 
    \begin{itemize}
        \item Quantum systems are inherently probabilistic
        \item A measurement can ony be made once
        \item We can only measure the position or momentum, but not both
    \end{itemize}
\end{frame}
\begin{frame}{Quantum Tomography}
    Quantum tomography provides a solution to this problem.\medbreak
    Key steps:
    \begin{enumerate}
        \item Replicate the initial state of the system multiple times
        \item Measure each clone once
        \item Calculate the empirical probabilities
        \item Estimate the quantum state with any appropriate method
    \end{enumerate}

\end{frame}

\begin{frame}{Quantum Tomography: mathematical description (1)}
    The Born rule states that
    \begin{equation}
        p(m) = \tr(\rho P_m)
    \end{equation}
    with
    \begin{itemize}
        \item $P_m$ the projector matrix associated to the eigenvalue $m$ of an \textit{observable} O
        \item $p(m)$ the probability of occurence  of $m$
        \item $\rho$ the \textit{density matrix} representing the quantum state
        \begin{itemize}
            \item positive semi-definite
            \item Hermitian ($\rho = \rho^\dagger$)
            \item trace$(\rho)=1$
        \end{itemize}
    \end{itemize}
\end{frame}
\begin{frame}{Quantum Tomography: mathematical description (2)}
    If we flatten the matrices
    \begin{equation}
        A = \begin{bmatrix}
            \vec P_1\\
            \vec P_2\\
            \vec P_3\\
            \vdots
        \end{bmatrix}
        \quad\vec{\rho} = \begin{bmatrix}
            \rho_{11} \\
            \rho_{12} \\
            \rho_{13} \\
            \vdots
        \end{bmatrix}
    \end{equation}

    then we can estimate $\rho$ by solving the resulting system of equations 
    \begin{equation}
        A\vec{\rho} = \hat p
    \end{equation}
\end{frame}
\begin{frame}{Existing methods}
    \begin{itemize}
        \item Direct methods: \begin{equation}
            \hat \rho = (A^TA)^{-1}A^T \hat p
        \end{equation}
        \item Optimization-based methods: \begin{equation}
            \hat \rho = \text{argmin}_{\vec\rho} ||A \vec\rho - \hat p||
        \end{equation} 
        \item Pauli basis expansion:\begin{equation}
            \hat \rho = \sum_{b\in\{I,x,y,z\}^n} \rho_b \sigma_b
        \end{equation}
        \item Bayesian methods, and in particular MCMC methods
        \begin{equation}
        \hat \rho = \frac{1}{N}\sum_{i=1}^N \nu_i \quad \text{with } \nu_i \sim \pi(\nu|\mb D)
        \end{equation}
    \end{itemize}
\end{frame}
\begin{frame}{Existing methods: our focus in this thesis}
    \begin{itemize}
        % \semitransp{
        \item<0> Direct methods: \begin{equation}
            \hat \rho = (A^TA)^{-1}A^T \hat p
        \end{equation}
        \item<0> Optimization-based methods: \begin{equation}
            \hat \rho = \text{argmin}_{\vec\rho} ||A \vec\rho - \hat p||
        \end{equation} 
        \item<0> Pauli basis expansion:
        \begin{equation}
            \hat \rho = \sum_{b\in\{I,x,y,z\}^n} \rho_b \sigma_b
        \end{equation}
        % }
        \item<1> Bayesian methods, and in particular MCMC methods
        \begin{equation}
        \hat \rho = \frac{1}{N}\sum_{i=1}^N \nu_i \quad \text{with } \nu_i \sim \pi(\nu|\mb D)
        \end{equation}
    \end{itemize}
\end{frame}

\begin{frame}{Markov chain Monte Carlo methods}
    \tb{Context}: We are working in the Bayesian framework:
    \begin{equation}
        \pi (\nu|\mb D) \propto \mathcal{L}(\mb D|\nu) \pi(\nu)
    \end{equation}
    Markov chain Monte Carlo (MCMC) methods allow us to \textit{sample} from $\pi (\nu|\mb D)$.\medbreak
    They build a Markov chain of samples $\nu_1, \nu_2, \dots$ where at equilibrium
    \begin{equation}
        f(x) =\pi (\nu|\mb D)
    \end{equation}
    with $f(x)$ the equilibrium distribution of the Markov chain.\medbreak
    Then, the density matrix is approximated as 
    \begin{equation}
        \hat \rho = \frac{1}{N}\sum_{i=1}^N \nu_i \quad \text{with } \nu_i \sim \pi(\nu|\mb D)
    \end{equation}
\end{frame}
\begin{frame}{The Metropolis-Hastings algorithm}
    The Metropolis-Hastings algorithm is one of the most common MCMC algorithms.\medbreak
    Given a first sample $\nu^{(0)}$ and until $t=T$:
    \begin{enumerate}
        \item Generate a candidate $\nu^* \sim q(\nu|\nu^{(t-1)})$
        \item Set $\nu^ {(t)} = \begin{cases}
            \nu^{*} \hspace{1.08cm} \text{with prob. } \alpha(\nu^*, \nu^{(t-1)})\\
            \nu^{(t-1)} \hspace{0.5cm} \text{with prob. } 1-\alpha(\nu^*, \nu^{(t-1)})
        \end{cases}$
    \end{enumerate}
    with \begin{equation}
        \alpha(\nu^*, \nu^{(t-1)}) = \frac{\pi(\nu^*|\mb D) q( \nu^{(t-1)}|\nu^*)}{\pi(\nu^{(t-1)}|\mb D) q( \nu^{*}|\nu^{(t-1)})}
    \end{equation}
\end{frame}
\begin{frame}{Prob-estimator}
    Introduced in MA17, it uses Metropolis-Hastings to , 
\end{frame}
\end{document}

